\documentclass[12pt]{article}
\usepackage{fullpage}
\usepackage{amsmath,amsfonts,amsthm}

% these are compressed lists to help fit into a 1 page limit
\newenvironment{enumerate*}%
  {\vspace{-2ex} \begin{enumerate} %
     \setlength{\itemsep}{-1ex} \setlength{\parsep}{0pt}}%
  {\end{enumerate}}
 
\newenvironment{itemize*}%
  {\vspace{-2ex} \begin{itemize} %
     \setlength{\itemsep}{-1ex} \setlength{\parsep}{0pt}}%
  {\end{itemize}}
 
\newenvironment{description*}%
  {\vspace{-2ex} \begin{description} %
     \setlength{\itemsep}{-1ex} \setlength{\parsep}{0pt}}%
  {\end{description}}

\DeclareMathOperator*{\E}{\mathbb{E}}
\let\Pr\relax
\DeclareMathOperator*{\Pr}{\mathbb{P}}

\newcommand{\inprod}[1]{\left\langle #1 \right\rangle}
\newcommand{\eqdef}{\mathbin{\stackrel{\rm def}{=}}}

\newtheorem{theorem}{Theorem}
\newtheorem{lemma}{Lemma}


% intersection

% union

% difference

% subset

% proper subset

% set laws w/ empty sets

% Associative laws

% distributive laws

% absorption laws

% demorgan's laws

% venn diagram, universe/ complement

% disjoint

% partition

% pairwise disjoint

% cardinality

% countable, uncountable
\author{Srikar Mylavarapu}
\title{Set Theory Introduction}

\begin{document}

\maketitle

Set theory is the study of sets. Which beg the question, what is a set?

A set is a collection of distinguishable elements. The elements are called either \emph{members} or \emph{elements}. Fundamentally, the following are sets.

\begin{theorem}

\end{theorem}


\begin{proof}
For the sake of contradiction suppose $\sqrt{2}$ is rational. Write $\sqrt{2} = a/b$ with $a,b$ positive integers with gcd $1$. Then $2 = a^2/b^2$, so $a = 2k$ is even. Then $2 = 4k^2/b^2$ so that $b = 2k^2$, implying $b$ is even. This contradicts that $a,b$ have gcd $1$.
\end{proof}

Some random facts in a list:

\begin{itemize*}
\item Compared with the ``itemize'' environment in \LaTeX, itemize$^*$ has smaller separation between bullet points.
\item The $n$th Catalan number is $C_n \eqdef \frac{1}{n+1} \binom{2n}{n}$.
\item If $\pi(x)$ is the number of primes less than or equal to $x$, then
$$\lim_{x\rightarrow\infty} \frac{\pi(x)}{x/\ln(x)} = 1 . $$
\item $$ \sum_{\substack{1\le i\le 2n\\i\ \mathrm{even}}} i = \frac{2n(n+1)}2 .$$
\end{itemize*}

Have fun on your problem sets.

\end{document}